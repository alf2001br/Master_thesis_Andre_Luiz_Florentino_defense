\begin{abstract}

\glsunsetall


The autonomous vehicle market is experiencing significant growth, with indications of transitioning from the "trough of disillusionment" to the "slope of enlightenment" on the Gartner hype cycle chart. This trajectory presents a promising opportunity to generate substantial revenue, potentially amounting to billions of dollars within the upcoming decade. The fundamental technologies encompassing extensive data analytics, computational capabilities, and sensor fusion techniques have already been established, and all stakeholders in this industry are persistently exploring novel approaches to enhance the overall perception of end users in terms of safety and trustworthiness. The main objective of this study is to investigate the feasibility of enhancing the perception abilities of autonomous vehicles by enabling them to recognize and categorize environmental sounds and merge its output in the sensor fusion network of the vehicle's architecture, resulting in specific responses either to the vehicle itself or the occupants. The methodology involved selecting relevant sound classes associated with the context of this study, evaluate the accuracy of several machine learning classifiers and neural network models, deploy the best-performing classifier to an embedded device, test the overall proposal using the realistic conditions of a regular passenger car and assess the potential enhancement of the vehicle’s safety, decision-making, and overall autonomy.

\glsresetall

\keywords{Environmental sound recognition, Autonomous vehicle, Embedded system, Feature extraction.}
\end{abstract}

\begin{resumo}

O mercado de veículos autônomos vem apresentando um crescimento significativo, com indicações de transição do "vale da desilusão" para a "inclinação da iluminação" no gráfico \textit{Gartner hype cycle}. Este movimento representa uma perspectiva promissora para a geração de receitas significativas, possivelmente na ordem de bilhões de dólares nos próximos dez anos. As tecnologias essenciais que abrangem análise extensa de dados, capacidades computacionais avançadas e técnicas de fusão de sensores já foram estabelecidas e todas as partes interessadas nesta indústria estão constantemente explorando novas abordagens para aprimorar a percepção global dos usuários finais em termos de segurança e confiabilidade. O objetivo central desta pesquisa é avaliar a viabilidade de aprimorar as capacidades perceptivas de veículos autônomos, ao permitir o reconhecimento e a categorização de sons ambientais, bem como a integração dessas informações na rede de fusão de sensores da arquitetura do veículo, resultando em respostas específicas tanto para o controle do veículo em si como para seus ocupantes. A metodologia consistiu na seleção de classes de sons pertinentes, relacionadas ao contexto específico desta pesquisa, na avaliação da acurácia de diversos classificadores de aprendizado de máquina e modelos de redes neurais, na implementação do classificador de melhor desempenho em um sistema embarcado, na realização de testes da abordagem proposta em um ambiente de condução realista e na avaliação do potencial aprimoramento da segurança, tomada de decisões e autonomia geral do veículo.

\palavraschave{Reconhecimento de som ambiental, Veículo autônomo, Sistema embarcado, Extração de características.}
\end{resumo}